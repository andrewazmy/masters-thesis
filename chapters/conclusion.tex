\chapter{Conclusion \& Future work}
The PID controller, which is the simplest controller is able to complete the whole track without crashing, which proves the simplicity of the problem. However, there is a significant difference in the driving performance between the controllers. The MPC has a clear edge when it comes to minimizing the cost function while maintaining smooth and realistic driving behavior. Although the PID controller yields lower RMSE than the CNN, the observed steering angle is erratic and unrealistic. Implementing such a controller on an electric steering system would lead to a hardware failure if not coupled to a smoothing block such as a low-pass filter, which can eventually hinder its performance. The CNN has the highest RMSE value, however, considering its unique system topology, with only the camera image as the provided input, it has proven itself to be the most practical among the three controllers despite its poor performance. A valid explanation of the oscillating response for the CNN in the start and end of the track, which are straight segments of the road, can be clearly associated to the data filtering step, in which straight driving data is clipped to remove any potential bias.